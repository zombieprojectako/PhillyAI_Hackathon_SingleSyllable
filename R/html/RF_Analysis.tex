\documentclass[]{article}
\usepackage{lmodern}
\usepackage{amssymb,amsmath}
\usepackage{ifxetex,ifluatex}
\usepackage{fixltx2e} % provides \textsubscript
\ifnum 0\ifxetex 1\fi\ifluatex 1\fi=0 % if pdftex
  \usepackage[T1]{fontenc}
  \usepackage[utf8]{inputenc}
\else % if luatex or xelatex
  \ifxetex
    \usepackage{mathspec}
  \else
    \usepackage{fontspec}
  \fi
  \defaultfontfeatures{Ligatures=TeX,Scale=MatchLowercase}
\fi
% use upquote if available, for straight quotes in verbatim environments
\IfFileExists{upquote.sty}{\usepackage{upquote}}{}
% use microtype if available
\IfFileExists{microtype.sty}{%
\usepackage{microtype}
\UseMicrotypeSet[protrusion]{basicmath} % disable protrusion for tt fonts
}{}
\usepackage[margin=1in]{geometry}
\usepackage{hyperref}
\hypersetup{unicode=true,
            pdftitle={Random Forest, Cycle Prediction},
            pdfborder={0 0 0},
            breaklinks=true}
\urlstyle{same}  % don't use monospace font for urls
\usepackage{color}
\usepackage{fancyvrb}
\newcommand{\VerbBar}{|}
\newcommand{\VERB}{\Verb[commandchars=\\\{\}]}
\DefineVerbatimEnvironment{Highlighting}{Verbatim}{commandchars=\\\{\}}
% Add ',fontsize=\small' for more characters per line
\usepackage{framed}
\definecolor{shadecolor}{RGB}{248,248,248}
\newenvironment{Shaded}{\begin{snugshade}}{\end{snugshade}}
\newcommand{\KeywordTok}[1]{\textcolor[rgb]{0.13,0.29,0.53}{\textbf{#1}}}
\newcommand{\DataTypeTok}[1]{\textcolor[rgb]{0.13,0.29,0.53}{#1}}
\newcommand{\DecValTok}[1]{\textcolor[rgb]{0.00,0.00,0.81}{#1}}
\newcommand{\BaseNTok}[1]{\textcolor[rgb]{0.00,0.00,0.81}{#1}}
\newcommand{\FloatTok}[1]{\textcolor[rgb]{0.00,0.00,0.81}{#1}}
\newcommand{\ConstantTok}[1]{\textcolor[rgb]{0.00,0.00,0.00}{#1}}
\newcommand{\CharTok}[1]{\textcolor[rgb]{0.31,0.60,0.02}{#1}}
\newcommand{\SpecialCharTok}[1]{\textcolor[rgb]{0.00,0.00,0.00}{#1}}
\newcommand{\StringTok}[1]{\textcolor[rgb]{0.31,0.60,0.02}{#1}}
\newcommand{\VerbatimStringTok}[1]{\textcolor[rgb]{0.31,0.60,0.02}{#1}}
\newcommand{\SpecialStringTok}[1]{\textcolor[rgb]{0.31,0.60,0.02}{#1}}
\newcommand{\ImportTok}[1]{#1}
\newcommand{\CommentTok}[1]{\textcolor[rgb]{0.56,0.35,0.01}{\textit{#1}}}
\newcommand{\DocumentationTok}[1]{\textcolor[rgb]{0.56,0.35,0.01}{\textbf{\textit{#1}}}}
\newcommand{\AnnotationTok}[1]{\textcolor[rgb]{0.56,0.35,0.01}{\textbf{\textit{#1}}}}
\newcommand{\CommentVarTok}[1]{\textcolor[rgb]{0.56,0.35,0.01}{\textbf{\textit{#1}}}}
\newcommand{\OtherTok}[1]{\textcolor[rgb]{0.56,0.35,0.01}{#1}}
\newcommand{\FunctionTok}[1]{\textcolor[rgb]{0.00,0.00,0.00}{#1}}
\newcommand{\VariableTok}[1]{\textcolor[rgb]{0.00,0.00,0.00}{#1}}
\newcommand{\ControlFlowTok}[1]{\textcolor[rgb]{0.13,0.29,0.53}{\textbf{#1}}}
\newcommand{\OperatorTok}[1]{\textcolor[rgb]{0.81,0.36,0.00}{\textbf{#1}}}
\newcommand{\BuiltInTok}[1]{#1}
\newcommand{\ExtensionTok}[1]{#1}
\newcommand{\PreprocessorTok}[1]{\textcolor[rgb]{0.56,0.35,0.01}{\textit{#1}}}
\newcommand{\AttributeTok}[1]{\textcolor[rgb]{0.77,0.63,0.00}{#1}}
\newcommand{\RegionMarkerTok}[1]{#1}
\newcommand{\InformationTok}[1]{\textcolor[rgb]{0.56,0.35,0.01}{\textbf{\textit{#1}}}}
\newcommand{\WarningTok}[1]{\textcolor[rgb]{0.56,0.35,0.01}{\textbf{\textit{#1}}}}
\newcommand{\AlertTok}[1]{\textcolor[rgb]{0.94,0.16,0.16}{#1}}
\newcommand{\ErrorTok}[1]{\textcolor[rgb]{0.64,0.00,0.00}{\textbf{#1}}}
\newcommand{\NormalTok}[1]{#1}
\usepackage{graphicx,grffile}
\makeatletter
\def\maxwidth{\ifdim\Gin@nat@width>\linewidth\linewidth\else\Gin@nat@width\fi}
\def\maxheight{\ifdim\Gin@nat@height>\textheight\textheight\else\Gin@nat@height\fi}
\makeatother
% Scale images if necessary, so that they will not overflow the page
% margins by default, and it is still possible to overwrite the defaults
% using explicit options in \includegraphics[width, height, ...]{}
\setkeys{Gin}{width=\maxwidth,height=\maxheight,keepaspectratio}
\IfFileExists{parskip.sty}{%
\usepackage{parskip}
}{% else
\setlength{\parindent}{0pt}
\setlength{\parskip}{6pt plus 2pt minus 1pt}
}
\setlength{\emergencystretch}{3em}  % prevent overfull lines
\providecommand{\tightlist}{%
  \setlength{\itemsep}{0pt}\setlength{\parskip}{0pt}}
\setcounter{secnumdepth}{0}
% Redefines (sub)paragraphs to behave more like sections
\ifx\paragraph\undefined\else
\let\oldparagraph\paragraph
\renewcommand{\paragraph}[1]{\oldparagraph{#1}\mbox{}}
\fi
\ifx\subparagraph\undefined\else
\let\oldsubparagraph\subparagraph
\renewcommand{\subparagraph}[1]{\oldsubparagraph{#1}\mbox{}}
\fi

%%% Use protect on footnotes to avoid problems with footnotes in titles
\let\rmarkdownfootnote\footnote%
\def\footnote{\protect\rmarkdownfootnote}

%%% Change title format to be more compact
\usepackage{titling}

% Create subtitle command for use in maketitle
\newcommand{\subtitle}[1]{
  \posttitle{
    \begin{center}\large#1\end{center}
    }
}

\setlength{\droptitle}{-2em}

  \title{Random Forest, Cycle Prediction}
    \pretitle{\vspace{\droptitle}\centering\huge}
  \posttitle{\par}
    \author{}
    \preauthor{}\postauthor{}
    \date{}
    \predate{}\postdate{}
  

\begin{document}
\maketitle

The goal of this project is to be able to predict the onset of a
ovulation cycle, given the termination point (date) of the prior
ovulation cycle, i.e.~the actual length of a cycle, given the past
history of he patient. The valuation of this process is measure by how
much been a prediction could be over a naive estimate. The naive
estimate, ignore all other attributes is based upon the mean cycle
length, which based on the current data set is 28.75 days.

\begin{Shaded}
\begin{Highlighting}[]
\CommentTok{# Libraries }
\KeywordTok{library}\NormalTok{(readr)}
\KeywordTok{library}\NormalTok{(randomForest)}
\end{Highlighting}
\end{Shaded}

\begin{verbatim}
## Warning: package 'randomForest' was built under R version 3.5.3
\end{verbatim}

\begin{verbatim}
## randomForest 4.6-14
\end{verbatim}

\begin{verbatim}
## Type rfNews() to see new features/changes/bug fixes.
\end{verbatim}

\begin{Shaded}
\begin{Highlighting}[]
\CommentTok{# Script to predict end dates }
\KeywordTok{set.seed}\NormalTok{(}\DecValTok{1234}\NormalTok{)}
\end{Highlighting}
\end{Shaded}

\subsection{Step 1, Load the Data Set}\label{step-1-load-the-data-set}

\begin{Shaded}
\begin{Highlighting}[]
\NormalTok{data <-}\StringTok{ }\KeywordTok{read_csv}\NormalTok{(}\StringTok{"~/GitHub/PhillyAI_Hackathon_SingleSyllable/data_out/model_sample_02.csv"}\NormalTok{)}
\end{Highlighting}
\end{Shaded}

\begin{verbatim}
## Parsed with column specification:
## cols(
##   user_id = col_double(),
##   cycle_start_date = col_date(format = ""),
##   acne = col_double(),
##   backache = col_double(),
##   bloating = col_double(),
##   cramp = col_double(),
##   diarrhea = col_double(),
##   dizzy = col_double(),
##   headache = col_double(),
##   mood = col_double(),
##   nausea = col_double(),
##   sore = col_double(),
##   key = col_double(),
##   est_cycle = col_double(),
##   TARGET_act_cycle = col_double(),
##   est_period = col_double(),
##   act_period = col_double(),
##   prior_period_end_date = col_date(format = "")
## )
\end{verbatim}

\subsection{Step 2, Calculate the naive
Estimate}\label{step-2-calculate-the-naive-estimate}

\begin{Shaded}
\begin{Highlighting}[]
\CommentTok{# Naive model }
\NormalTok{ave_cycle =}\StringTok{ }\KeywordTok{mean}\NormalTok{(data}\OperatorTok{$}\NormalTok{TARGET_act_cycle)}
\KeywordTok{print}\NormalTok{(ave_cycle)}
\end{Highlighting}
\end{Shaded}

\begin{verbatim}
## [1] 28.7503
\end{verbatim}

\subsection{Step 3, Created the train/test data sets
(Partition)}\label{step-3-created-the-traintest-data-sets-partition}

\begin{Shaded}
\begin{Highlighting}[]
\CommentTok{# create test and training sets }
\NormalTok{inTrain <-}\StringTok{ }\KeywordTok{sample}\NormalTok{(}\DecValTok{1}\OperatorTok{:}\KeywordTok{nrow}\NormalTok{(data), }\KeywordTok{nrow}\NormalTok{(data) }\OperatorTok{*}\StringTok{ }\FloatTok{0.85}\NormalTok{) }\CommentTok{# select 85% of the items }
\NormalTok{train <-}\StringTok{ }\NormalTok{data[inTrain, ]}
\NormalTok{test <-}\StringTok{ }\NormalTok{data[}\OperatorTok{-}\NormalTok{inTrain, ]}
\KeywordTok{rm}\NormalTok{(data, inTrain)}
\end{Highlighting}
\end{Shaded}

\subsection{Step 4, Build Model and Make
Predicitions}\label{step-4-build-model-and-make-predicitions}

\begin{Shaded}
\begin{Highlighting}[]
\NormalTok{cols <-}\StringTok{ }\KeywordTok{c}\NormalTok{(}\StringTok{'user_id'}\NormalTok{, }\StringTok{'acne'}\NormalTok{, }\StringTok{'backache'}\NormalTok{,}
          \StringTok{'bloating'}\NormalTok{, }\StringTok{'cramp'}\NormalTok{, }\StringTok{'diarrhea'}\NormalTok{, }
          \StringTok{'dizzy'}\NormalTok{, }\StringTok{'headache'}\NormalTok{, }\StringTok{'mood'}\NormalTok{, }\StringTok{'nausea'}\NormalTok{, }
          \StringTok{'sore'}\NormalTok{, }\StringTok{'est_cycle'}\NormalTok{, }\StringTok{'est_period'}\NormalTok{, }\StringTok{'act_period'}\NormalTok{)}
\NormalTok{rf <-}\StringTok{ }\KeywordTok{randomForest}\NormalTok{(train}\OperatorTok{$}\NormalTok{TARGET_act_cycle }\OperatorTok{~}\StringTok{ }\NormalTok{., }\DataTypeTok{data=}\NormalTok{train[,cols], }\DataTypeTok{ntree=}\DecValTok{200}\NormalTok{)}
\NormalTok{predicted <-}\StringTok{ }\KeywordTok{predict}\NormalTok{(rf, test[,cols])}
\end{Highlighting}
\end{Shaded}

\subsection{Step 5, Determine Accuracy of RF model using
Root-Mean-Square-Error
(RMSE)}\label{step-5-determine-accuracy-of-rf-model-using-root-mean-square-error-rmse}

\begin{Shaded}
\begin{Highlighting}[]
\CommentTok{# actual results }
\NormalTok{results <-}\StringTok{ }\KeywordTok{as.data.frame}\NormalTok{(}\KeywordTok{cbind}\NormalTok{(test}\OperatorTok{$}\NormalTok{TARGET_act_cycle, predicted))}
\NormalTok{results}\OperatorTok{$}\NormalTok{err =}\StringTok{ }\NormalTok{(results}\OperatorTok{$}\NormalTok{V1}\OperatorTok{-}\NormalTok{results}\OperatorTok{$}\NormalTok{predicted)}\OperatorTok{^}\DecValTok{2}
\NormalTok{mse_rf =}\StringTok{ }\KeywordTok{mean}\NormalTok{(results}\OperatorTok{$}\NormalTok{err)}
\KeywordTok{print}\NormalTok{(}\KeywordTok{sqrt}\NormalTok{(mse_rf))}
\end{Highlighting}
\end{Shaded}

\begin{verbatim}
## [1] 4.511861
\end{verbatim}

\subsection{Step 6, Determine Accuracy of naive Model using
RMSE}\label{step-6-determine-accuracy-of-naive-model-using-rmse}

\begin{Shaded}
\begin{Highlighting}[]
\CommentTok{# Naive Results }
\NormalTok{results2 <-}\StringTok{ }\KeywordTok{as.data.frame}\NormalTok{(}\KeywordTok{cbind}\NormalTok{(test}\OperatorTok{$}\NormalTok{TARGET_act_cycle))}
\NormalTok{results2}\OperatorTok{$}\NormalTok{predicted =}\StringTok{ }\NormalTok{ave_cycle}
\NormalTok{results2}\OperatorTok{$}\NormalTok{err =}\StringTok{ }\NormalTok{(results2}\OperatorTok{$}\NormalTok{V1}\OperatorTok{-}\NormalTok{results2}\OperatorTok{$}\NormalTok{predicted)}\OperatorTok{^}\DecValTok{2}
\NormalTok{mse_ne =}\StringTok{ }\KeywordTok{mean}\NormalTok{(results2}\OperatorTok{$}\NormalTok{err)}
\KeywordTok{print}\NormalTok{(}\KeywordTok{sqrt}\NormalTok{(mse_ne))}
\end{Highlighting}
\end{Shaded}

\begin{verbatim}
## [1] 5.091454
\end{verbatim}

\subsection{Step 7, Compare accuracy of RF and Naive
models}\label{step-7-compare-accuracy-of-rf-and-naive-models}

\begin{Shaded}
\begin{Highlighting}[]
\CommentTok{# Improvement }
\NormalTok{net_improvement =}\StringTok{ }\NormalTok{(mse_ne }\OperatorTok{-}\StringTok{ }\NormalTok{mse_rf)}\OperatorTok{/}\NormalTok{mse_ne}
\KeywordTok{print}\NormalTok{(net_improvement)}
\end{Highlighting}
\end{Shaded}

\begin{verbatim}
## [1] 0.214714
\end{verbatim}

\subsection{Conclusion}\label{conclusion}

Based upon the evidence provided, it appears that the RF model improved
accuracy over a naive model by approximately 21.47\%.


\end{document}
